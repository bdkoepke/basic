\section{Numbers}
\subsection{The Integers}
\definition{Positive Integers $\mathbb{Z}^+$:  $1, 2, 3, 4, \dots$}
\definition{Origin: 0}
\definition{Natural Numbers $\mathbb{N}^+$: $0, 1, 2, 3, \dots$}
\definition{Negative Integers $\mathbb{Z}^-$: $-1, -2, -3, \dots$}
\theorem{$0 + a = a + 0 = a$}
\theorem{$a + (-a) = -a + a = 0$}
\definition{Additive Inverse of $a$: $-a$}
\theoremstyle{plain}
\newtheorem{addition}{Addition}[section]
\begin{addition}
Commutativity:
\[
\forall a, b \in \mathbb{Z}, a + b = b + a
\]
\end{addition}
\begin{addition}
Associativity:
\[
\forall a, b, c \in \mathbb{Z}, (a + b) + c = a + (b + c)
\]
\end{addition}
\theorem{$\forall a, b \in \mathbb{Z}, a + b = 0 \implies (b = -a) \land (a = -b)$}
\theorem{$\forall a \in \mathbb{Z}, a = -(-a)$}
\theorem{$\forall a, b \in \mathbb{Z}, -(a + b) = -a - b$}
\addition{$\forall a, b \in \mathbb{Z}^+ \implies a + b \in \mathbb{Z}^+$}
\addition{$\forall a, b \in \mathbb{Z}^- \implies a + b \in \mathbb{Z}^-$}
\subsection{Rules for Multiplication}
\newtheorem{multiplication}{Multiplication}[section]
\begin{multiplication}
Commutativity:
\[
\forall a, b \in \mathbb{Z}, ab = ba
\]
\end{multiplication}
\begin{multiplication}
Associativity:
\[
\forall a, b, c \in \mathbb{Z}, (ab)c = a(bc)
\]
\end{multiplication}
\theorem{$\forall a \in \mathbb{Z}, 1a = a$ and $\forall a \in \mathbb{Z}, 0a = 0$}
\begin{multiplication}
Distributivity
\[
\forall a, b, c \in \mathbb{Z}, a(b + c) = ab + ac
\]
and
\[
\forall a, b, c \in \mathbb{Z}, (b + c)a = ba + bc
\]
\end{multiplication}
\theorem{$\forall a \in \mathbb{Z}, (-1)a = -a$}
\theorem{$\forall a, b \in \mathbb{Z}, -(ab) = (-a) b$}
\theorem{$\forall a, b \in \mathbb{Z}, -(ab) = a(-b)$}
\theorem{$\forall a, b \in \mathbb{Z}, (-a)(-b) = ab$}
\definition{$n$-th power of $a$ is $a^n = aaa...a$ ($n$ times)}
\theorem{$\forall m, n \in \mathbb{Z}^+, a^{m + n} = a^ma^n$}
\theorem{$\forall m, n \in \mathbb{Z}^+, (a^m)^n = a^{mn}$}
\begin{example}
\begin{align*}
(ab)^n &= a^nb^n \\
(ab)^n &= abab...ab \\
 &= aa...abb...b \\
 &=a^nb^n
\end{align*}
\end{example}
\begin{multiplication}
\[
(a + b)^2 = a^2 + 2ab + b^2
\]
\[
(a - b)^2 = a^2 - 2ab + b^2
\]
\[
(a + b)(a - b) = a^2 - b^2
\]
\end{multiplication}
\subsection{Even and Odd Integers; Divisibility}
\definition{$\forall n \in \mathbb{Z}^+, \exists m \in \mathbb{Z}^+, n = 2m \implies n$ is even}
\definition{$\forall n \in \mathbb{Z}, \exists m \in \mathbb{N}^+ \text{ (including 0)}, n = 2m + 1 \implies n$ is odd}
\begin{theorem}
Let $E(x)$ be the predicate for $x$ is even and let $O(x)$ be the predicate for $x$ is odd.
\[
\forall a, b \in \mathbb{Z}^+, E(a) \land E(b) \implies E(a + b)
\]
\[
\forall a, b \in \mathbb{Z}^+, E(a) \land O(b) \implies O(a + b)
\]
\[
\forall a, b \in \mathbb{Z}^+, O(a) \land E(b) \implies O(a + b)
\]
\[
\forall a, b \in \mathbb{Z}^+, O(a) \land O(b) \implies E(a + b)
\]
\end{theorem}
\begin{theorem}
\[
\forall a \in \mathbb{Z}^+, E(a) \implies E(a^2)
\]
\[
\forall a \in \mathbb{Z}^+, O(a) \implies O(a^2)
\]
\end{theorem}
\begin{corollary}
\[
\forall a \in \mathbb{Z}^+, E(a^2) \implies E(a)
\]
\[
\forall a \in \mathbb{Z}^+, O(a^2) \implies O(a)
\]
\end{corollary}
\definition{$d$ divides $n$ or $n$ is divisible by $d$ if $n = dk$ for some integer $k$}
\subsection{Rational Numbers}
\definition{Rational Number $\mathbb{Q}$:  can be written in the form $\frac{m}{n}$, where $m, n \in \mathbb{Z}$ and $n \neq 0$}
\newtheorem{rational}{Rational}[section]
\begin{rational}
Rule for cross-multiplying.
\[
\forall m, n, r, s \in \mathbb{Z}, n \neq 0, s \neq 0 \implies \cfrac{m}{n} = \cfrac{r}{s} \iff ms = rn
\]
\end{rational}
\begin{rational}
Cancellation rule for fractions
\[
\forall a, m, n \in \mathbb{Z}, a \neq 0, n \neq 0 \implies \cfrac{am}{an} = \cfrac{m}{n}
\]
\end{rational}
\begin{theorem}
Any positive rational number has an expression as a fraction in lowest form
\end{theorem}
\begin{rational}
Addition rule for rational numbers
\[
\cfrac{a}{d} + \cfrac{b}{d} = \cfrac{a + b}{d}
\]
\end{rational}
\begin{rational}
General case of addition rule for rational numbers
\[
\cfrac{m}{n} + \cfrac{r}{s} = \cfrac{ms + rn}{ns}
\]
\end{rational}
\corollary{The sum of positive rational numbers is also positive}
\corollary{$\forall a \in \mathbb{R} \implies 0 + a = a + 0 = a$}
\corollary{Addition of rational numbers satisfies the properties of commutativity and associativity}
\begin{rational}
Multiplication of rational numbers
\[
\cfrac{m}{n}*\cfrac{r}{s} = \cfrac{mr}{ns}
\]
\end{rational}
\corollary{$\forall k \in \mathbb{Z}^+, a^k = \left(\cfrac{m}{n}\right)^k = \cfrac{m^k}{n^k}$}
\theorem{There is no positive rational number whose square is 2}
\definition{Irrational: a number which is not rational}
\corollary{For any rational number $a$ we have $1a = a$ and $0a = 0$. Furthermore, multiplication is associative, commutative, and distributive with respect to addition.}
\begin{rational}
Rational numbers satisfy the property
\[
\forall a \in \mathbb{R}, a \neq 0 \implies \exists a^{-1}, a^{-1}a = aa^{-1} = 1
\]
\end{rational}
\begin{rational}
Cross-multiplication
\[
\forall a, b, c, d \in \mathbb{R}, b \neq 0, d \neq 0 \implies \cfrac{a}{b} = \cfrac{c}{d} \implies ad = bc, ad = bc \implies \cfrac{a}{b} = \cfrac{c}{d}
\]
\end{rational}
\begin{rational}
Cancellation law for multiplication
\[
\forall a, b, c \in \mathbb{R} a \neq 0, ab = ac \implies b = c
\]
\end{rational}
\begin{rational}
Common denominator
\[
\cfrac{a}{b} + \cfrac{c}{d} = \cfrac{ad + bc}{bd}
\]
\end{rational}
