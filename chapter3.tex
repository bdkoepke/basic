\section{Chapter 3: Real Numbers}
\subsection{Addition and Multiplication}
\begin{addition}
Addition is commutative and associative, meaning that
\[
\forall a, b, c \in \mathbb{R}, a + b = b + a, a + (b + c) = (a + b) + c
\]
Furthermore
\[
0 + a = a
\]
To each real number $a$ there is an associated $a$ denoted by $-a$ such that
\[
a + (-a) = 0
\]
\end{addition}
\begin{multiplication}
Properties of multiplication, multiplication is commutative and associative, meaning that
\[
\forall a, b, c \in \mathbb{R}, ab = ba, a(bc) = (ab)c
\]
Furthermore
\[
1a = a, 0a = 0
\]
Multiplication is distributive with respect to addition meaning that
\[
a(b + c) = ab + ac, (b + c)a = ba + ca
\]
\end{multiplication}
\begin{corollary}
For real numbers
\[
(a + b)^2 = a^2 + 2ab + b^2
\]
\[
(a - b)^2 = a^2 - 2ab + b^2
\]
\[
(a + b)(a - b) = a^2 - b^2
\]
\end{corollary}
\begin{corollary}
There is also the existence of a multiplicative inverse
\[
\forall a \in \mathbb{R}, a \neq 0 \exists a^{-1} \mid a^{-1}a = aa^{-1} = 1
\]
\end{corollary}
\subsection{Real Numbers: Positivity}
\theoremstyle{plain}
\newtheorem{positivity}{Positivity}[section]
\positivity{If $a, b$ are positive then so are the product $ab$ and the sum $a + b$}
\positivity{If $a \in \mathbb{R}$ then either $a$ is positive or $a = 0$ or $-a$ is positive, and these possibilities are mutually exclusive.}
\positivity{If $a$ is positive and $b$ is negative, then $ab$ is negative}
\positivity{If $a$ is negative and $b$ is negative, then $ab$ is positive}
\positivity{If $a$ is positive, then $1/a$ is positive}
\positivity{If $a$ is negative, then $1/a$ is negative}
\corollary{$\forall a \in \mathbb{R}, a > 0 \implies \exists b \in \mathbb{R} \mid b^2 = a$}
\corollary{$\forall x, y \in \mathbb{R}, x^2 = y^2 \implies x = y $ or $x -y$}
\definition{Absolute value of $x$ is $|x| = \sqrt{x^2}$}
\subsection{Powers and Roots}
\theoremstyle{plain}
\newtheorem{powers}{Powers}[section]
\definition{The product of $a$ with itself $n$ times is $\forall n \in \mathbb{Z}^+ a \in \mathbb{R} a^n$}
\corollary{$\forall m, n \in \mathbb{Z}^+ a^{m+n} = a^ma^n$}
\corollary{$\forall a \in \mathbb{R}^+, n \in \mathbb{Z}^+ \exists r \in \mathbb{R}^+ \mid r^n = a$}
\definition{The $n^{\text{th}}$ root of $a$ is $a^{1/n}$ or $\sqrt[n]{a}$}
\powers{$\forall a, b \in \mathbb{R}^+ (ab)^{1/n} = a^{1/n}b^{1/n}$}
\begin{corollary}
Let $a$ be a positive number. To each rational number $x$ we can associate a positive number denoted by $a^x$, which is the $n^{\text{th}}$ power of $a$ when $x$ is a positive integer $n$, the $n^{\text{th}}$ root of $a$ when $x = 1/n$, and satisfying the following conditions:
\powers{$\forall x, y \in \mathbb{Q} a^{x + y} = a^xa^y$}
\powers{$\forall x, y \in \mathbb{Q} (a^x)^y = a^{xy}$}
\powers{$\forall a, b \in \mathbb{R}^+ (ab)^x = a^xb^x$}
\end{corollary}
\corollary{$\forall a \in \mathbb{Q} a^0 = 1$}
\corollary{$\forall a \in \mathbb{R} x \in \mathbb{R}^+ a^{-x} = \cfrac{1}{a^x}$}
\corollary{$\forall a \in \mathbb{R} m, n \in \mathbb{Z}^+ a^{m/n} = (a^m)^{1/n} = (a^{1/n})^m$}
\subsection{Inequalities}
\theoremstyle{plain}
\newtheorem{inequalities}{Inequalities}[section]
\inequalities{$\forall a \in \mathbb{R}^+ \implies a > 0$}
\inequalities{$\forall a, b \in \mathbb{R} a - b > 0 \implies a > b$}
\inequalities{$\forall a \in \mathbb{R} -a > 0 \implies a < 0$}
\inequalities{$\forall a, b \in \mathbb{R} a > b \implies b < a$}
\inequalities{$\forall a, b \in \mathbb{R} a \geq b \implies a > b \lor a = b$}
\inequalities{$\forall a, b, c \in \mathbb{R} a > b \land b > c \implies a > c$}
\inequalities{$\forall a, b, c \in \mathbb{R} a > b \land c > 0 \implies ac > bc$}
\inequalities{$\forall a, b, c \in \mathbb{R} a > b \land c < 0 \implies ac < bc$}
\definition{open interval: $\forall x, a, b \in \mathbb{R} \mid a < x < b$}
\definition{closed interval: $\forall x, a, b \in \mathbb{R} \mid a \leq x \leq b$}
\definition{half open or half closed: $\forall x, a, b \in \mathbb{R} \mid a \leq x < b, a < x \leq b$}
\definition{infinite interval: $\forall x, a \in \mathbb{R} \mid x < a \lor x > a \lor x \leq a \lor x \geq a$}
